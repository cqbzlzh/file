\documentclass{ctexbeamer}        % 文档类beamer的汉化版本

\usefonttheme{serif}              % 使用衬线字体
\usefonttheme{professionalfonts}  % 数学公式字体
\usepackage{amsmath}
\usepackage{mathtools}
\usepackage{hyperref}
\usepackage{listings}
\usepackage{fontspec} % 定制字体
\usepackage{xcolor} % 定制颜色
\usepackage{hyperref}
\definecolor{mygreen}{rgb}{0,0.6,0}
\definecolor{mygray}{rgb}{0.5,0.5,0.5}
\definecolor{mymauve}{rgb}{0.58,0,0.82}
\lstset{ %
backgroundcolor=\color{white},      % choose the background color
basicstyle=\footnotesize\ttfamily,  % size of fonts used for the code
columns=fullflexible,
tabsize=4,
breaklines=true,               % automatic line breaking only at whitespace
captionpos=b,                  % sets the caption-position to bottom
commentstyle=\color{mygreen},  % comment style
escapeinside={\%*}{*)},        % if you want to add LaTeX within your code
keywordstyle=\color{blue},     % keyword style
stringstyle=\color{mymauve}\ttfamily,  % string literal style
frame=none,
rulesepcolor=\color{red!20!green!20!blue!20},
% identifierstyle=\color{red},
language=c++,
}

%% --> 主题和色彩风格
\usetheme{Frankfurt}       %主题和色彩可以自由搭配
\usecolortheme{orchid}     %主题和色彩的样式可参见网页 https://mpetroff.net/files/beamer-theme-matrix/

\newcommand{\qbinom}[2]{\genfrac{[}{]}{0pt}{}{#1}{#2}}
\newcommand{\stirling}[2]{\genfrac\{ \} {0pt}{}{#1}{#2}}
\newcommand{\lcm}{\operatorname{lcm}}

\begin{document}

%% --> 导言页
%
\title{同余方程}
\author{Quack}
\date{\today}
\frame{\titlepage}

%% --> 目录结构
\begin{frame}{目录}         %自动生成目录
  \tableofcontents[hideallsubsections]
\end{frame}

%% --> 正式内容开始
\section{线性同余方程}

\begin{frame}{线性同余方程}

\begin{definition}[线性同余方程]
    $$ax \equiv b \pmod m$$
\end{definition}

对于一个线性同余方程$ax \equiv b \pmod m$,将其转化成二元一次不定方程$ax+my=b$,然后用之前我们讲过的方法去解即可。时间复杂度$O(\log m)$。

\end{frame}

\AtBeginSection[]{\frame{\tableofcontents[currentsection,hideallsubsections]}}

\section{逆元}

\begin{frame}{逆元}

\begin{definition}[逆元]
    对正整数$a$,$a$在模$m$意义下的逆元$a^{-1}$满足$aa^{-1} \equiv 1 \pmod m$。
\end{definition}

求一个数的逆元,可以转化成二元一次不定方程$ax+my=1$,然后用之前我们讲过的方法去解即可。时间复杂度$O(\log m)$。

当$m$是质数的时候,由费马小定理,$a^{m-1}\equiv 1 \pmod m$,所以$a$和$a^{m-2}$互为逆元。可以用快速幂得到。时间复杂度$O(\log m)$。

\end{frame}

\begin{frame}{逆元}

\begin{example}[多个数的逆元]
    给出正整数$a_1,\cdots,a_n$,求这些数在模质数$p$意义下的逆元。要求复杂度$O(n)$。
\end{example}
\pause
设前缀积$prod_i=\prod_{j=1}^i a_i$。可以$O(n)$递推求出$1\sim n$每个位置的前缀积。

设前缀积的逆$iprod_i=prod_i^{-1}=prod_n^{-1}\prod_{j=i+1}^n a_i$。在求出$prod_n$后,花$O(\log p)$的时间求出$iprod_n$。然后可以$O(n)$递推求出$1\sim n$每个位置的前缀积的逆。

$a_i$的逆$ia_i=a_i^{-1}=iprod_i prod_{i-1}$。所以在求出每个位置的前缀积和前缀积的逆后,可以$O(n)$求出$1\sim n$每个位置的逆元。总复杂度$O(n+\log p)$。

\end{frame}

\section{线性同余方程组}
\begin{frame}{线性同余方程组}
\begin{definition}[线性同余方程组]
    有$n$个方程$x\equiv a_i \pmod{p_i}$,$p_i$两两互质,求$x$。
\end{definition}
\begin{theorem}[中国剩余定理,CRT]
    设$P=\prod_{i=1}^n p_i,w_i=\frac{P}{p_i}$,则方程组的解为$x \equiv \sum_{i=1}^n a_iw_i inv(w_i,p_i) \pmod P$,其中,$inv(w_i,p_i)$表示$w_i$在模$p_i$意义下的逆。
\end{theorem}
不难验证这确实是方程组的一个解。直接按照公式计算,复杂度$O(n\log p)$。
\end{frame}

\begin{frame}{线性同余方程组}
\begin{theorem}[中国剩余定理,CRT]
    设$P=\prod_{i=1}^n p_i,w_i=\frac{P}{p_i}$,则方程组的解为$x \equiv \sum_{i=1}^n a_iw_i inv(w_i,p_i) \pmod P$,其中,$inv(w_i,p_i)$表示$w_i$在模$p_i$意义下的逆。

    并且,CRT给出的是模$P$意义下的唯一解。
\end{theorem}

\begin{proof}
    设$x_1$和$x_2$都是线性同余方程组的解。则对所有的$i$,有$x_1\equiv x_2 \pmod{p_i}$,即$p_i|(x_1-x_2)$,所以$\lcm(p_1,\cdots,p_n)|(x_1-x_2)$,又因为$p_i$两两互质,所以$\lcm(p_1,\cdots,p_n)=P$。
\end{proof}
\end{frame}

\begin{frame}{线性同余方程组}
\begin{example}[exCRT]
    考虑求方程组$x\equiv a_1 \pmod{p_1}$和$x\equiv a_2 \pmod{p_2}$的解($p_1$和$p_2$不互质)。
\end{example}
设$d=\gcd(p_1,p_2)$,那么必须$a_1\equiv a_2 \pmod d$。然后方程组的解一定可以表示为$wd+(a_1 \bmod d)$。

所以有$wd+(a_1 \bmod d) \equiv a_1 \pmod{p_1} $,移项有$wd \equiv a_1-(a_1 \bmod d) \pmod{p_1} $,即$w \equiv \frac{a_1}{d} \pmod{\frac{p_1}{d}}$。

类似地,有$w \equiv \frac{a_2}{d} \pmod{\frac{p_2}{d}}$。这两个同余方程的模数就互质了。可以用CRT求出$w$,进而得到原方程组的解。用之前的证明思路,同样可以证明这个方程的解是唯一的。

注意到这样做每次会把两个方程合并成一个模数是它们lcm的方程。如果有多个这样的方程可以两两合并。
\end{frame}

\begin{frame}{线性同余方程组}
\begin{theorem}[扩展欧拉定理]
    若$b\ge \varphi(m)$,则$a^b\equiv a^{b \bmod \varphi(m)+\varphi(m)} \pmod m$。
\end{theorem}
注意到``扩展''之处是不要求$a$和$m$互质。
\begin{proof}
    设$m$的质因子分解为$m=\prod p_i^{a_i}$,由之前讲的欧拉函数的性质,$\varphi(p_i^{a_i})|\varphi(m)$。

    考虑CRT,如果左右在模所有的$p_i^{a_i}$都相等,则它们相等。

    当$p_i|a$时,由于$b \ge \varphi(m)\ge \varphi(p_i^{a_i}) \ge a_i$,于是左右两边都为$0$。

    否则说明$\gcd(p_i,a)=1$,可以用欧拉定理,两边显然是相等的。
\end{proof}
\end{frame}

\section{高次同余初步}

\begin{frame}{高次同余初步}
\begin{example}[bsgs]
    设$\gcd(a,m)=1$,考虑求方程$a^x\equiv b \pmod m$的解。
\end{example}
令$B=\lceil \sqrt{m} \rceil$,把$(ba^0,0),(ba^1,1),\cdots,(ba^{B-1},B-1)$全部插入hash表里面。

然后令$d=a^B$,计算$d^1,d^2,\cdots,d^B$,然后查询hash表里面有没有$d^i$。如果有,设对应的指数是$k$,那就说明$a^{iB}\equiv ba^k \pmod m$,由于$a,m$互质,所以有$a^{iB-k} \equiv b \pmod m$。那么就解出了方程的一个解$x=iB-k$。如果需要所有的解,这里还要继续枚举下去。如果枚举完了都没有,那就无解了。

时间复杂度$O(\sqrt{m})$。
\end{frame}

\begin{frame}{高次同余初步}
\begin{example}[exbsgs]
    当$\gcd(a,m)\neq 1$时,考虑求方程$a^x\equiv b \pmod m$的解。
\end{example}
我们可以考虑从$x$个$a$中拿出$c$个$a$,与$b$和$m$消去公因子,直到$a$和$m'$互质为止。

一旦互质了,方程就是$va^{x-c}=b' \pmod{m'}$。$v$是拿出$c$个$a$消去公因子后剩下的东西,$b',m'$是消去公因子的$b,m$。这时还要求$v$关于$m'$的逆元,方程变为$a^{x-c}=b'inv(v,m')\pmod{m'}$。此时就可以用bsgs做了,答案为bsgs的答案加$c$。时间复杂度$O(\sqrt{m})$。

在实现的时候,可以不求$v$的逆元。bsgs里面新增一个参数$v$,然后哈希表比的时候一边是$va^{iB}$,一边是$ba^j$。

注意,有可能在消的过程中方程两边就已经相等了,所以必须消去一次就特判一次方程两边(就是$v$和$b'$)是不是相等了。如果相等,直接返回此时的$c$。
\end{frame}

\begin{frame}[fragile]
\frametitle{高次同余初步}

根据上述算法流程,不难写出代码:
\begin{block}{exbsgs}
\begin{lstlisting}[language={c++},
                   numbers=left]
ll BSGS(ll a,ll b,ll p,ll v=1){
    ll m=ceil(sqrt(p)),val=1,d;
    for(int i=0;i<m;++i){
        ht.insert(b*val%p,i);
        val=val*a%p;
    }
    d=v*val%p;
    for(int i=1;i<=m;++i){
        int res=ht.query(d);
        if(res!=-1)return i*m-res;
        d=d*val%p;
    }
    return -1;
}
\end{lstlisting}
\end{block}
\end{frame}

\begin{frame}[fragile]
\frametitle{高次同余初步}

根据上述算法流程,不难写出代码:
\begin{block}{exbsgs}
\begin{lstlisting}[language={c++},
                   numbers=left]
ll EXBSGS(ll a,ll b,ll p){
    ll t,c=0,v=1;
    while((t=gcd(a,p))!=1){
        if(b%t)return -1;
        p/=t;b/=t;
        v=v*a/t%p; // 注意这个地方p先除了再对v取模
        ++c;if(b==v)return c;
    }
    ll ret=BSGS(a,b,p,v);
    return ret!=-1?ret+c:ret;
}
\end{lstlisting}
\end{block}
\end{frame}

\section{练习}

\begin{frame}{线性同余方程}
\begin{example}[accoders2241]
    \url{http://www.accoders.com/problem.php?id=2241}
\end{example}
\pause
设跳了$t$步,不难写出方程$mt+x\equiv nt+y \pmod L$,然后移项,解就可以了。
\end{frame}

\begin{frame}{线性同余方程组}
\begin{example}[accoders2244]
    \url{http://www.accoders.com/problem.php?id=2244}
\end{example}
crt的模板题。
\end{frame}

\begin{frame}{扩展欧拉定理}
\begin{example}[accoders3704]
    给出$p$,求$2^{2^{2^{2^{\cdots}}}} \bmod p$。

    数据范围:$p \le 10^7$。
\end{example}
\pause
扩展欧拉定理的一个应用是求无穷层的幂,如这个题。

记$f(p)=2^{2^{2^{2^{\cdots}}}} \bmod p$,则$f(p)\equiv 2^{f(\varphi(p))+\varphi(p)} \pmod p$。

于是我们不断递归至$\varphi(p)=1$,此时$f(\varphi(p))=0$。
\end{frame}

\begin{frame}{bsgs}
\begin{example}[accoders10371 \& 9100]
    \url{http://www.accoders.com/problem.php?id=10371}

    \url{http://www.accoders.com/problem.php?id=9100}
\end{example}
分别是bsgs和exbsgs的模板题。
\end{frame}

\begin{frame}{bsgs}
\begin{example}[accoders10376]
    \url{http://www.accoders.com/problem.php?id=10376}
\end{example}
这说明了写不带求逆的bsgs的好处。
\end{frame}

\section{extra}

\begin{frame}{exlucas}

\begin{theorem}[exlucas]
	求$\binom{n}{m} \bmod q$,$q$为合数。
\end{theorem}
首先,我们把$q$分解质因数,那么问题就转化成求$\binom{n}{m} \bmod{p^a}$,对每个质数的幂最后再用CRT合并。

由于阶乘中可能有质因子$p$,所以不能直接求逆。而是应该求下面两个问题:

\begin{itemize}
    \item $n!$中$p$的次数为多少?
    \item $n!$中把所有$p$的质因子提出来之后,$\frac{n!}{p^l} \bmod p^a$的值?
\end{itemize}
\end{frame}

\begin{frame}{exlucas}

对于第一个问题,$n!$中$p$的次数等于$n!$中含有因子$p$的数的个数加$n!$中含有因子$p^2$的数的个数,以此类推,一直加下去。比较好算。

对于第二个问题,首先把$n!$中$p$的倍数提出来,一共有$\frac{n}{p}$个,都提一个$p$以后,就是一个规模为$\frac{n}{p}$的子问题。由于$p$是质数,所以$n!$中非$p$的倍数就和$p$互质。在模$p^a$意义下会循环$\frac{n}{p^a}$次,还有一个长为$n \bmod{p^a}$的不完整的循环,这个不完整的循环暴力算。循环节内部的元素都是小于$p^a$且和$p^a$互质的元素,这相当于求模$p^a$意义下缩系元素之积。有结论(hdu4910),当$p>2$或$p=2,a\le 2$时,答案就是$p^a-1$,当$p=2,a\ge 3$时,答案就是$1$。当然你也可以暴力算。
\end{frame}

\begin{frame}[plain]    %%谢谢
	\vspace{0.4\textheight}
	\begin{center}
		\Huge\color{blue}\bfseries Thanks!
	\end{center}
\end{frame}

\end{document}