\documentclass{ctexbeamer}        % 文档类beamer的汉化版本

\usefonttheme{serif}              % 使用衬线字体
\usefonttheme{professionalfonts}  % 数学公式字体
\usepackage{amsmath}
\usepackage{mathtools}
\usepackage{hyperref}
\usepackage{listings}
\usepackage{fontspec} % 定制字体
\usepackage{xcolor} % 定制颜色
\usepackage{hyperref}
\definecolor{mygreen}{rgb}{0,0.6,0}
\definecolor{mygray}{rgb}{0.5,0.5,0.5}
\definecolor{mymauve}{rgb}{0.58,0,0.82}
\lstset{ %
backgroundcolor=\color{white},      % choose the background color
basicstyle=\footnotesize\ttfamily,  % size of fonts used for the code
columns=fullflexible,
tabsize=4,
breaklines=true,               % automatic line breaking only at whitespace
captionpos=b,                  % sets the caption-position to bottom
commentstyle=\color{mygreen},  % comment style
escapeinside={\%*}{*)},        % if you want to add LaTeX within your code
keywordstyle=\color{blue},     % keyword style
stringstyle=\color{mymauve}\ttfamily,  % string literal style
frame=none,
rulesepcolor=\color{red!20!green!20!blue!20},
% identifierstyle=\color{red},
language=c++,
}

%% --> 主题和色彩风格
\usetheme{Frankfurt}       %主题和色彩可以自由搭配
\usecolortheme{orchid}     %主题和色彩的样式可参见网页 https://mpetroff.net/files/beamer-theme-matrix/

\newcommand{\qbinom}[2]{\genfrac{[}{]}{0pt}{}{#1}{#2}}
\newcommand{\stirling}[2]{\genfrac\{ \} {0pt}{}{#1}{#2}}
\newcommand{\lcm}{\operatorname{lcm}}

\begin{document}

%% --> 导言页
%
\title{数论函数初步}
\author{Quack}
\date{\today}
\frame{\titlepage}

%% --> 目录结构
\begin{frame}{目录}         %自动生成目录
  \tableofcontents[hideallsubsections]
\end{frame}

%% --> 正式内容开始
\section{数论函数}

\begin{frame}{数论函数}

\begin{definition}[数论函数]
    在全体正整数(或者整数)上定义的函数称作数论函数。
\end{definition}

\begin{definition}[积性]
    积性:若$\gcd(a,b)=1$,则$f(ab)=f(a)f(b)$。
\end{definition}

\end{frame}

\AtBeginSection[]{\frame{\tableofcontents[currentsection,hideallsubsections]}}

\begin{frame}{数论函数}

\begin{example}[积性数论函数]
    $1(x)=1,Id(x)=x,I(x)=[x=1]$;

    欧拉函数$\varphi(x)$。
\end{example}

\end{frame}

\begin{frame}{数论函数}

\begin{definition}[狄利克雷卷积]
    数论函数$f(n)$和$g(n)$的狄利克雷卷积$h(n)$定义为:
    $$h(n)=\sum_{d|n}f(d)g(\frac{n}{d})$$

    记作$h=f \ast g$。
\end{definition}

\begin{theorem}
    两个积性函数的狄利克雷卷积也是积性函数。
\end{theorem}

\end{frame}

\begin{frame}{数论函数}

\begin{proof}
    \begin{align*}
        &h(xy)=\sum_{d|xy} f(d)g(\frac{xy}{d})=\sum_{d_1|x}\sum_{d_2|y} f(d_1d_2)g(\frac{xy}{d_1d_2})\\
        =&\sum_{d_1|x}f(d_1)g(\frac{x}{d_1})\sum_{d_2|y} f(d_2)g(\frac{y}{d_2})=h(x)h(y)
    \end{align*}
\end{proof}

\end{frame}

\begin{frame}{数论函数}

\begin{theorem}
    两个积性函数的对应位置相乘也是积性函数。
\end{theorem}

很显然,证明略。

\end{frame}

\begin{frame}{数论函数}

所以我们可以定义更多的数论函数,以及用卷积描述它们之间的关系:
\begin{example}[数论函数]
    $d(x)=\sum_{d|n}1$,即$d=1 \ast 1$;

    $n = \sum_{d|n} \varphi(d)$,即$Id=\varphi \ast 1$。
\end{example}

\end{frame}

\begin{frame}{数论函数}

下面研究卷积的性质。
\begin{theorem}[卷积运算律]
    交换律:设有两个积性函数$f,g$,则$f \ast g = g\ast f$。

    结合律:设有两个积性函数$f,g,h$,则$(f \ast g) \ast h = f \ast (g\ast h)$。
\end{theorem}

交换律的证明很显然。

结合律的证明可以把式子改写成矩阵形式,然后用矩阵的结合律来证明。应该大力推式子也可以。

\end{frame}

\begin{frame}{数论函数}

\begin{theorem}[单位元]
    $I$是单位元:对任意的数论函数$f$,$f \ast I = I \ast f = f$。
\end{theorem}

\begin{proof}
    $$f(n)=\sum_{d|n}f(d)I(\frac{n}{d})$$
\end{proof}

\end{frame}

\begin{frame}{数论函数}

\begin{definition}[狄利克雷逆]
    若$f \ast g = I$,则数论函数$f,g$互为彼此的狄利克雷逆。
\end{definition}

\end{frame}

\section{莫比乌斯反演}
\begin{frame}{莫比乌斯反演}
\begin{definition}[莫比乌斯函数]
    定义莫比乌斯函数$\mu(n)$。
    
    当$n=1$时,$\mu(n)=1$。

    当$n$是square-free number时,设$n$的质因数分解有$k$项,则$\mu(n)=(-1)^k$。

    否则,$\mu(n)=0$。
\end{definition}
不难验证$\mu$也是积性函数。
\end{frame}

\begin{frame}{莫比乌斯反演}
\begin{theorem}
    $I = \mu \ast 1$,即$\mu$和$1$互为彼此的逆。
\end{theorem}
\begin{proof}
    设$n$的不同质因子有$k$个,分别为$p_1,\cdots,p_k$,那么有
    \begin{align*}
        &\sum_{d|n}\mu(d)=\sum_{i=1}^k \sum_{\text{k个选i个,积为c}} \mu(c)\\
        =&\sum_{i=1}^k \sum_{\text{k个选i个}} (-1)^i=\sum_{i=1}^k \binom{k}{i} (-1)^i=(1-1)^k\\
        =&[k=0]=I(n)
    \end{align*}
\end{proof}
\end{frame}

\begin{frame}{莫比乌斯反演}
\begin{definition}[莫比乌斯变换]
    设数论函数$f$,称$F=f \ast 1$为莫比乌斯变换。
\end{definition}
\begin{definition}[莫比乌斯反演]
    设数论函数$F$,称$f=F \ast \mu$为莫比乌斯反演。
\end{definition}
\begin{example}
    $d=1 \ast 1$,即$d \ast \mu = 1$;

    $Id=\varphi \ast 1$,即$Id \ast \mu = \varphi$。
\end{example}
\end{frame}

\section{线性筛}

\begin{frame}{厄拉多塞筛法}
\begin{example}[厄拉多塞筛法]
    $O(n\log \log n)$判断$2 \sim n$的所有数是否是质数。
\end{example}
有一张写了$2\sim n$这些数的表,$2$是质数,我们把表中其他所有$2$的倍数划去。然后$3$没有被划去,说明$3$是质数,把表中其他所有$3$的倍数划去。以此类推。

时间复杂度$O(n+\frac{n}{2}+\frac{n}{3}+\frac{n}{5}+\frac{n}{7}+\cdots)=O(n\log\log n)$。
\end{frame}

\begin{frame}{线性筛}
\begin{example}[线性筛]
    $O(n)$判断$2 \sim n$的所有数是否是质数。
\end{example}
厄拉多塞筛法在筛$30=2\times 3 \times 5$这样的数时,会被$2,3,5$重复筛。

线性筛的思路是,保证每个数只被它的最小的质因子筛去。
\end{frame}

\begin{frame}{线性筛}
\begin{example}[线性筛]
    $O(n)$判断$2 \sim n$的所有数是否是质数。
\end{example}
厄拉多塞筛法在筛$30=2\times 3 \times 5$这样的数时,会被$2,3,5$重复筛。

线性筛的思路是,保证每个数只被它的最小的质因子筛去。
\end{frame}

\begin{frame}[fragile]
\frametitle{线性筛}

我们先给出代码:
\begin{block}{线性筛}
\begin{lstlisting}[language={c++},
                   numbers=left]
for(int i = 2; i <= n; i++){
    if(isp[i] == 0) pr[++cnt] = i;
    for(int j = 1; j <= cnt && pr[j] * i <= n; j++){  
        isp[i * pr[j]] = 1;  
        if(i % pr[j] == 0) break;  
    }  
}
\end{lstlisting}
\end{block}
保证每个数只被它的最小的质因子筛去,代码中体现在i \% pr[j] == 0。

pr数组中的质数是递增的,当$pr[j]|i$时,$pr[j]|pj[j+1]i$,那么$pj[j+1]i$这个合数应该被$pr[j]$这个更小的质数筛掉。

另外,当$pr[j]|i$时,$pr[j]$是$i$的最小质因子。否则$pr[j]$是$pr[j]i$的最小质因子。
\end{frame}

\begin{frame}{线性筛}
\begin{example}[线性筛]
    $O(n)$求出某些积性函数在$1 \sim n$处的所有取值。
\end{example}
我们沿用线性筛的过程,考虑这个问题。

首先,线性筛筛出质数的时候,我们需要求这个积性函数在质数处的取值。

其次,在for循环中,设$k=pr[j]$,当$k|i$时,由上面的讨论,$k$是$i$的最小质因子,设$k$在$i$中的幂次为$a$,那么$f(ki)=\frac{f(k^{a+1})}{f(k^a)}f(i)$。

否则,因为$k$是质数,那么$\gcd(k,i)=1$,那么$f(ki)=f(k)f(i)$。
\end{frame}

\begin{frame}[fragile]
\frametitle{线性筛}
\begin{example}[线性筛欧拉函数]
    $O(n)$求出欧拉函数在$1 \sim n$处的所有取值。
\end{example}
由于$\frac{\varphi(k^{a+1})}{\varphi(k^a)}=k$,为一个定值,所以线性筛欧拉函数很容易。
\begin{block}{线性筛欧拉函数}
\begin{lstlisting}[language={c++},
                   numbers=left]
for(int i = 2; i <= n; i++){
    if(isp[i] == 0) {pr[++cnt] = i; phi[i] = i - 1;}
    for(int j = 1; j <= cnt && pr[j] * i <= n; j++){  
        isp[i * pr[j]] = 1;  
        if(i % pr[j] == 0) {
            phi[i * pr[j]] = phi[i] * pr[j];
            break;
        }
        phi[i * pr[j]] = phi[i] * (pr[j] - 1);
    }  
}
\end{lstlisting}
\end{block}
\end{frame}

\begin{frame}[fragile]
\frametitle{线性筛}
\begin{example}[线性筛莫比乌斯函数]
    $O(n)$求出莫比乌斯函数在$1 \sim n$处的所有取值。
\end{example}
由于$\frac{\mu(k^{a+1})}{\mu(k^a)}=0$,为一个定值,所以也很容易。
\begin{block}{线性筛莫比乌斯函数}
\begin{lstlisting}[language={c++},
                   numbers=left]
for(int i = 2; i <= n; i++){
    if(isp[i] == 0) {pr[++cnt] = i; mu[i] = -1;}
    for(int j = 1; j <= cnt && pr[j] * i <= n; j++){  
        isp[i * pr[j]] = 1;  
        if(i % pr[j] == 0) {
            mu[i * pr[j]] = 0;
            break;
        }
        mu[i * pr[j]] = mu[i] * -1;
    }  
}
\end{lstlisting}
\end{block}
\end{frame}

\begin{frame}[fragile]
\frametitle{线性筛}
\begin{example}[线性筛约数个数]
    $O(n)$求出约数个数函数$\tau$在$1 \sim n$处的所有取值。
\end{example}
由于$\frac{\tau(k^{a+1})}{\tau(k^a)}=\frac{a+2}{a+1}$,不为一个定值,所以在筛的过程中还要维护$i$的最小质因子的次数。
\end{frame}

\begin{frame}[fragile]
\frametitle{线性筛}
\begin{block}{线性筛约数个数}
\begin{lstlisting}[language={c++},
                   numbers=left]
for(int i = 2; i <= n; i++){
    if(isp[i] == 0) {pr[++cnt] = i; tau[i] = 2; g[i] = 2;}
    for(int j = 1; j <= cnt && pr[j] * i <= n; j++){  
        isp[i * pr[j]] = 1;  
        if(i % pr[j] == 0) {
            tau[i * pr[j]] = (g[i] + 1) * tau[i] / g[i];
            g[i * pr[j]] = g[i] + 1;
            break;
        }
        tau[i * pr[j]] = tau[i] * 2;
        g[i * pr[j]] = 2;
    }
}
\end{lstlisting}
\end{block}
\end{frame}

\section{整数分块}

\begin{frame}{整数分块}

\begin{theorem}[整数分块]
    对任意的$i$,$\lfloor \frac{n}{i} \rfloor$只有$O(\sqrt{n})$种取值。
\end{theorem}

\begin{proof}
    当$1\le i \le \sqrt{n}$时,$\frac{n}{i}$有$O(\sqrt{n})$种取值。

    当$i > \sqrt{n}$时,$\frac{n}{i} \le \sqrt{n}$,也只有$O(\sqrt{n})$种取值。

    综上,只有$O(\sqrt{n})$种取值。
\end{proof}

\end{frame}

\begin{frame}[fragile]
\frametitle{整数分块}

\begin{example}[计算下取整分式的和式]
    计算$\sum_{i=1}^n \lfloor \frac{n}{i} \rfloor$。
\end{example}

由于$\lfloor \frac{n}{i} \rfloor$只有$O(\sqrt{n})$种取值,并且,使得$\lfloor \frac{n}{i} \rfloor$取相同取值的$i$也是一段一段的,所以我们只需要一段一段地计算即可。

\begin{block}{整数分块}
\begin{lstlisting}[language={c++},
                   numbers=left]
ll res = 0;
for(int l = 1, r; l <= n; l = r + 1) {
    r = n / (n / l);
    res += (r - l + 1) * (n / l);
}
\end{lstlisting}
\end{block}

\end{frame}

\section{练习}

\begin{frame}{莫比乌斯反演}
\begin{example}[accoders3465]
    给出$n,m$,求$2\sum_{i=1}^n\sum_{j=1}^m \gcd(i,j) - nm$。

    数据范围:$n,m \le 10^5$。
\end{example}
\pause
我们之前学欧拉函数的时候,能求$\sum_{i=1}^n\sum_{j=1}^n \gcd(i,j)$,但是当$n\neq m$时,就不能再用欧拉函数求了。

还是之前的套路,$\sum_{i=1}^n\sum_{j=1}^m \gcd(i,j)=\sum_{k=1}^{\min(n,m)}k\sum_{i=1}^n\sum_{j=1}^m [\gcd(i,j)=k]$。

设$a=\lfloor \frac{n}{k} \rfloor,b=\lfloor \frac{m}{k} \rfloor$,继续推:

$\sum_{i=1}^n\sum_{j=1}^m \gcd(i,j)=\sum_{k=1}^{\min(n,m)}k\sum_{i=1}^a\sum_{j=1}^b [\gcd(i,j)=1]$。
\end{frame}

\begin{frame}{莫比乌斯反演}
\begin{align*}
    &\sum_{k=1}^{\min(n,m)}k\sum_{i=1}^a\sum_{j=1}^b [\gcd(i,j)=1]=\sum_{k=1}^{\min(n,m)}k\sum_{i=1}^a\sum_{j=1}^b I(\gcd(i,j))\\
    =&\sum_{k=1}^{\min(n,m)}k\sum_{i=1}^a\sum_{j=1}^b \sum_{d|\gcd(i,j)} \mu(d)=\sum_{k=1}^{\min(n,m)}k\sum_{d=1}^{\min(a,b)}\sum_{i=1}^{\lfloor \frac{a}{d} \rfloor}\sum_{j=1}^{\lfloor \frac{b}{d} \rfloor}\mu(d)\\
    =&\sum_{k=1}^{\min(n,m)}k\sum_{d=1}^{\min(a,b)}\mu(d)\lfloor \frac{a}{d} \rfloor \lfloor \frac{b}{d} \rfloor
\end{align*}
到这里用整数分块已经可以做到$O(n\sqrt{n})$的复杂度,足以通过此题。
\end{frame}

\begin{frame}[fragile]
\frametitle{莫比乌斯反演}
\begin{block}{整数分块}
\begin{lstlisting}[language={c++},
                   numbers=left]
for(int k=1;k<=min(n,m);k++){
    int a=n/k,b=m/k;
    ll sum=0;
    for(int l=1,r;l<=min(a,b);l=r+1){
        r=min(a/(a/l),b/(b/l));
        sum+=1ll*(smu[r]-smu[l-1])*(a/l)*(b/l);
    }
    ans+=sum*k;
}
\end{lstlisting}
\end{block}
\end{frame}

\begin{frame}{莫比乌斯反演}
由答案的对称性,不妨设$n \le m$,然后我们继续推:
\begin{align*}
    &\sum_{k=1}^{n}k\sum_{d=1}^{a}\mu(d)\lfloor \frac{a}{d} \rfloor \lfloor \frac{b}{d} \rfloor=\sum_{k=1}^{n}k\sum_{d=1}^{\lfloor \frac{n}{k} \rfloor}\mu(d)\lfloor \frac{n}{kd} \rfloor \lfloor \frac{m}{kd} \rfloor \\
    \overset{T=kd}{=}& \sum_{T=1}^{n} \lfloor \frac{n}{T} \rfloor \lfloor \frac{m}{T} \rfloor \sum_{k|T} k\mu(\frac{T}{k})
\end{align*}
由于$Id \ast \mu = \varphi$,所以$\sum_{k|T} k\mu(\frac{T}{k})=\varphi(T)$,那么最终答案就是$\sum_{T=1}^{n} \lfloor \frac{n}{T} \rfloor \lfloor \frac{m}{T} \rfloor \varphi(T)$。这个式子在线性筛出$\varphi$及其前缀和的情况下可以用整数分块做到$O(\sqrt{n})$一次回答(用于应对多组询问的情况)。这样我们就得到了一种$O(n)-O(\sqrt{n})$的算法。
\end{frame}

\begin{frame}{莫比乌斯反演}
\begin{example}[accoders6724]
    给出$n,m$,求$\sum_{i=1}^n\sum_{j=1}^m [\gcd(i,j) \in P]$。

    数据范围:多组数据,$1 \le T \le 10^4$,$n,m \le 10^7$。
\end{example}
\pause
类似的推法,最后得到:
$$\sum_{T=1}^{n} \lfloor \frac{n}{T} \rfloor \lfloor \frac{m}{T} \rfloor \sum_{k|T,k\in P} \mu(\frac{T}{k})$$
设$g(T)=\sum_{k|T,k\in P} \mu(\frac{T}{k})$,看上去不是很好线性筛,不过可以厄拉多塞筛法在$O(n\log\log n)$内算出其前缀和,然后用整数分块回答询问。
\end{frame}

\begin{frame}[plain]    %%谢谢
	\vspace{0.4\textheight}
	\begin{center}
		\Huge\color{blue}\bfseries Thanks!
	\end{center}
\end{frame}

\end{document}